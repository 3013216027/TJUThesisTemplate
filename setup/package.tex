% !Mode:: "TeX:UTF-8"
%  Authors: 张井   Jing Zhang: prayever@gmail.com     天津大学2010级管理与经济学部信息管理与信息系统专业硕士生
%           余蓝涛 Lantao Yu: lantaoyu1991@gmail.com  天津大学2008级精密仪器与光电子工程学院测控技术与仪器专业本科生
%  Modified by: 郑冬健 Dongjian Zheng: zhengdongjian2013@gmail.com 天津大学2013级计算机科学与技术专业本科生

%%%%%%%%%% Package %%%%%%%%%%%%
\usepackage{lmodern}
\usepackage[T1]{fontenc}
\usepackage{graphicx}                       % 支持插图处理
\usepackage[a4paper,text={146.4true mm,239.2 true mm},top= 26.2true mm,left=31.8 true mm,head=6true mm,headsep=6.5true mm,foot=16.5true mm]{geometry}
                                            % 支持版面尺寸设置
\usepackage[squaren]{SIunits}               % 支持国际标准单位
\usepackage{titlesec}                       % 控制标题的宏包
\usepackage{titletoc}                       % 控制目录的宏包
\usepackage{fancyhdr}                       % fancyhdr宏包 支持页眉和页脚的相关定义
% \usepackage[UTF8]{ctex}                     % 支持中文显示
% \usepackage{xunicode}
\usepackage[AutoFakeBold=3,PunctStyle=hangmobanjiao]{xeCJK}
% \usepackage{CJKpunct}                       % 精细调整中文的标点符号
\usepackage{color}                          % 支持彩色
\usepackage{amsmath}                        % AMSLaTeX宏包 用来排出更加漂亮的公式
\usepackage{amssymb}                        % 数学符号生成命令
\usepackage[below]{placeins}                %允许上一个section的浮动图形出现在下一个section的开始部分,还提供\FloatBarrier命令,使所有未处理的浮动图形立即被处理
\usepackage{multirow}                       % 使用Multirow宏包,使得表格可以合并多个row格
\usepackage{diagbox}                        % 形如'\'的表格分栏
\usepackage{booktabs}                       % 表格,横的粗线;\specialrule{1pt}{0pt}{0pt}
\usepackage{longtable}                      % 支持跨页的表格。
\usepackage{tabularx}                       % 自动设置表格的列宽
\usepackage{subfigure}                      % 支持子图 %centerlast 设置最后一行是否居中
\usepackage[subfigure]{ccaption}            % 支持子图的中文标题
% \usepackage{caption}
\usepackage[sort&compress,numbers]{natbib}  % 支持引用缩写的宏包
% \usepackage[resetlabels]{multibib}          % 支持在中文译文中加入新的参考文献
\usepackage{enumitem}                       % 使用enumitem宏包,改变列表项的格式
\usepackage{calc}                           % 长度可以用+ - * / 进行计算
\usepackage{txfonts}                        % 字体宏包
\usepackage{bm}                             % 处理数学公式中的黑斜体的宏包
\usepackage[amsmath,thmmarks,hyperref]{ntheorem}  % 定理类环境宏包,其中 amsmath 选项用来兼容 AMS LaTeX 的宏包
\usepackage{zhnumber}                       % 提供将阿拉伯数字转换成中文数字的命令
\usepackage{indentfirst}                    % 首行缩进宏包
\usepackage{hypbmsec}                       % 用来控制书签中标题显示内容
\newcommand{\tabincell}[2]{\begin{tabular}{@{}#1@{}}#2\end{tabular}}
\usepackage{xcolor}
%支持代码环境
\usepackage{listings}
\lstset{numbers=left,
language=[ANSI]{C},
numberstyle=\tiny,
extendedchars=false,
showstringspaces=false,
breakatwhitespace=false,
breaklines=true,
captionpos=b,
keywordstyle=\color{blue!70},
commentstyle=\color{red!50!green!50!blue!50},
frame=shadowbox,
rulesepcolor=\color{red!20!green!20!blue!20}
}
%支持算法环境
\usepackage[boxed,ruled,lined]{algorithm2e}
\usepackage{algorithmic}

\usepackage{array}
\newcommand{\PreserveBackslash}[1]{\let\temp=\\#1\let\\=\temp}
\newcolumntype{C}[1]{>{\PreserveBackslash\centering}p{#1}}
\newcolumntype{R}[1]{>{\PreserveBackslash\raggedleft}p{#1}}
\newcolumntype{L}[1]{>{\PreserveBackslash\raggedright}p{#1}}

% 生成有书签的 pdf 及其生成方式。目录需要多次编译方可生成。
% 请使用xelatex + bibtex + xelatex + xelatex 编译。
\def\atemp{xelatex}\ifx\atemp\usewhat
\usepackage[unicode,
            pdfstartview=FitH,
            bookmarksnumbered=true,
            bookmarksopen=true,
            colorlinks=false,
            pdfborder={0 0 1},
            citecolor=blue,
            linkcolor=red,
            anchorcolor=green,
            urlcolor=blue,
            breaklinks=true
            ]{hyperref}
\fi

% tikz绘图,支持流程图、思维导图等
\usepackage{tikz}
\usetikzlibrary{arrows,shapes,chains}
% 设置颜色代号
\colorlet{lcfree}{green}
\colorlet{lcnorm}{blue}
\colorlet{lccong}{red}
% -------------------------------------------------
% 设置调试标志层
\pgfdeclarelayer{marx}
\pgfsetlayers{main,marx}
% 标记坐标点的宏定义。交换下面两个定义关闭。
\providecommand{\cmark}[2][]{%
  \begin{pgfonlayer}{marx}
    \node [nmark] at (c#2#1) {#2};
  \end{pgfonlayer}{marx}
  }
\providecommand{\cmark}[2][]{\relax}
% -------------------------------------------------
